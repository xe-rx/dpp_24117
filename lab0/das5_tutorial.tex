\documentclass[parskip]{scrartcl}

\usepackage{graphicx}
\usepackage{hyperref}
\usepackage{hyperref}
\usepackage{float}
\usepackage{amsmath,siunitx,booktabs}
\usepackage[margin=2.5cm]{geometry}

\renewcommand{\familydefault}{\sfdefault}

\titlehead{University of Amsterdam\\BSc Informatica}
\subject{Distributed and Parallel Programming}
\title{DAS5 Tutorial}
\subtitle{Becoming Familiar with the Supercomputing Infrastructure}
\author{dr. Adam Belloum}
\date{October 2023}

\begin{document}

\maketitle

In this first tutorial you will learn some basic features of the DAS5  supercomputer. This tutorial will focus on the details needed in the first three
assignments of the Distributed and Parallel Programming course, namely:

\begin{itemize}
\item How to set up your working environment.
\item The rules of using the DAS5 systems.
\item How to run a simple application using DAS5.
\item How to use queuing systems.
\item How to run simple CUDA programs on the DAS5 special nodes.
\end{itemize}

For more details about DAS5 consult the DAS5 web page: \url{https://www.cs.vu.nl/das5/} and the DAS5 instructions as published on Canvas.  

\section*{Questions}

\begin{enumerate}
\item DAS5 login and data copying
    \begin{enumerate}
    \item How do you log in on DAS5? What is needed to login from home/from outside the UvA? 
    \item Where should students change their password?
    \item How do you copy a file from your local file system to the DAS5? As an example, please copy the \texttt{hello\_cuda.cu} and \texttt{Makefile} files, provided on Canvas. How do you copy a folder from DAS5 to the local file system?  
    \end{enumerate}
    
\item Usage policy
    \begin{enumerate}
    \item What is the default run time for jobs?
    \item What is the maximum run-time for a job during the day?  
    \item How and when can one execute long-running programs?
    \item What are the permitted actions on the head node and on the regular nodes?  
    \item What are the consequences of not following the rules?
    \end{enumerate}

\item Job Execution
    \begin{enumerate}
    \item How is the Prun user interface used? Please follow the Prun/MPI example (\url{https://www.cs.vu.nl/das5/jobs.shtml}). Modify the example to run on 1 nodes with 1 process each. How do you know (based on the output(s)) you have succeeded to change the execution configuration? 
    \item Assume you have just compiled your application, into "assign1", in the current directory, and you are ready to execute it. What would be the command to run it on the headnode (which, as you know, you should never do)? What about running it on a single regular node?  
    \end{enumerate}
    
\item GPU computing 
    \begin{enumerate}
    \item How do you setup the environment for running CUDA programs?
    \item Use the files you copied (the \texttt{hello\_cuda.cu} and \texttt{Makefile} from above) to build the GPU application (just use \texttt{make}). Run this GPU application using prun. What is the output?   
    \end{enumerate}
\end{enumerate}

\end{document}

